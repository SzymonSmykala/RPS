\documentclass[fleqn]{article}

\usepackage[T1]{fontenc}

\usepackage{amscd}
\usepackage{amsfonts}
\usepackage{latexsym}
\usepackage{graphicx}
\usepackage{amsthm}

\usepackage[utf8]{inputenc}
\usepackage{amsmath}
\usepackage{amsfonts}
\usepackage{amssymb}
\usepackage{mathtools}
\usepackage{cancel}
\usepackage{calligra}
\usepackage{enumerate}

\def\B{\mathcal{B}}
\def\E{\mathcal{E}}
\def\F{\mathcal{F}}
\def\G{\mathcal{G}}
\def\O{\mathcal{O}}
\def\Q{\mathcal{Q}}
\def\P{\mathcal{P}}
\def\C{\mathcal{C}}
\def\U{\mathcal{U}}
\def\uni{\mathrm{unif}}
\def\Code{\mathrm{Code}}
\def\diag{\mathrm{diag}}
\def\Mah{\mathrm{Mah}}
\def\N{\mathbb{N}}
\def\Z{\mathbb{Z}}
\def\R{\mathbb{R}}
\def\Nor{\mathcal{N}}
\def\I{\mathrm{I}}
\def\f{\mathrm{f}}
\def\m{\mathrm{m}}

% definicje przydatnych potem komend

\def\C{{\mathbb C}}
\def\Z{{\mathbb Z}}
\def\N{{\mathbb N}}
\def\R{{\mathbb R}}
\def\Q{{\mathbb Q}}

\def\e{\varepsilon}
\def\w{\omega}
\def\dom{\mathrm{dom}}

\begin{document}

\title{Zadania z Rachunku Prawdopodobie�stwa \\ �wiczenia .. z dnia ...}
\date{}
\medskip

\noindent{\bf Zad. 1 (KO)}

B - wypadną 2 orły, $P(B) = \frac{1}{4}$. \\
a) A - wypadnie orzeł w pierwszym rzucie, $P(A) = \frac{1}{2}$; \\
   \[P(A|B) = \frac{1}{2}\]
b) A - w dowolnym rzucie wypadł orzeł, $P(A) = \frac{3}{4}$. \\
   \[P(A|B) =\frac{1}{3}\]

\medskip

\noindent{\bf Zad. 2 (KO)}
\medskip

Po rozdaniu dla pierwszysch 2 graczy zostaje 26 kart, w nich 5 trefli. Gracz E moze wylosowac swoje 13 kart na ${26 \choose 13}$. Żeby wsród wylosowanych kart mieć 3 trefli na ${5 \choose 3}\cdot{23 \choose 10}$\\
\[P = \frac{{5 \choose 3}\cdot{23 \choose 10}}{{26 \choose 13}}\]

\medskip

\noindent{\bf Zad. 3 (MM)}
\medskip

a) Nie sa.\\
$P(E_{1})=\frac{5}{36},P(F)=\frac{1}{6}, P(E_{1}\cap F)=\frac{1}{36}\\ P(E_{1}\cap F)\neq P(E_{1})P(F)$\\
b) Sa.\\
$P(E_{2})=\frac{1}{6}, P(E_{2}\cap F)=\frac{1}{36}$\\
$P(E_{2}\cap F)=P(E_{2})P(F)$\\

\medskip
\noindent{\bf Zad. 4 (MK)}
\medskip

0.8 - moduł nie przeszedł inspekcji

0.2 - moduł przeszedł inspekcję

$P(A) = 0.2 \cdot 0.95 = 0.19$ - moduł bez wad po inspekcji

$P(B) = 0.8 \cdot 0.7 = 0.56$ - moduł bez wad bez inspekcji

$P(C) = 0.2 - 0.19 = 0.01$ - moduł wadliwy po inspekcji

$P(D) = 0.8 - 0.56 = 0.24$ - moduł wadliwy bez inspekcji

$P(E) = \frac{P(C)}{P(C)+(P(D)}$ - szansa na to, aby moduł wadliwy był po inspekcji

$\frac{0.01}{0.25} = 0.04$

\medskip
\noindent{\bf Zad. 5 (MM)}
\medskip

b)\\
$n_{1}$ -ilosc kul w pierwszej urnie, $b_{1}$ -ilosc kul bialych w pierwszej urnie, $c_{1}$ -ilosc kul czarnych w pierwszej urnie, $n_{2}$, $b_{2}$, $c_{2}$ 
analogicznie tylko dla urny drugiej.\\
Mamy w sumie n losowan. Losowanie wyglada tak: Najpierw z Urny pierwszej losujemy kule i ja usuwamy z urny a potem z urny drugiej losujemy kule i ja wkladamy do 
urny pierwszej.\\
$B_{k}$ -Zdazenie ze w k-tym losowaniu z urny pierwszej wylosujemy kule biala\\
$C_{k}$ -Zdarzenie ze w k-tym losowaniu z urny pierwszej wylosujemy kule czarna\\
$\tilde{B}_{k}$ -Zdazenie ze w k-tym losowaniu z urny drugiej wylosujemy kule biala\\
$\tilde{C}_{k}$ -Zdarzenie ze w k-tym losowaniu z urny drugiej wylosujemy kule Czarna\\
$D_{k}$ - Zdarzenie ze w k-tym losowaniu z urny pierwszej wylosujemy te sama kule co w k-1-szym losowaniu z urny drugiej(wylosujemy te kule co wlasnie swiezo 
dodalismy do urny pierwszej)\\
$D_{k}^{-1}$ - Zdarzenie przeciwne do $D_{k}$: Zdarzenie ze w k-tym losowaniu z urny pierwszej wylosujemy inna kule niz te co w k-1-szym losowaniu z urny drugiej. 
Oczywiscie:\\$P(B_{k})+P(C_{k})=1=P(\tilde{B}_{k})+P(\tilde{C}_{k})$, dla kazdego k.\\
Chcemy znalesc ogolny wzor na $P(B_{k})$.\\
Obliczmy $P(\tilde{B}_{k})$:\\
Kulki w drugiej Urnie sa inteligentne i zanim zaczelismy je wyciagac ustawily się w kolejke do wyciagania. Ustalily to w sposob losowy Wiec $P(\tilde{B}_{k})$ 
Mozemy traktowac jako: Prawdobodobienstwo że na k-tym miejscu w kolejce ustawila sie kulka biala. Oczywiscie numer pozycji k, nie ma wplywu na wynik bo wszystkie 
pozycje w kolejce sa jednakowo uprzywilejowane. Prawdobodobienstwo że na k-tym miejscu w kolejce ustawila sie kulka biala wynosi $\frac{b_{2}}{n_{2}}$ bo z $n_{2}$ 
potencjalnych kul interesuja nas tylko $b_{2}$ bialych. Zatem:\\
$P(\tilde{B}_{k})=\frac{b_{2}}{n_{2}}$, analogicznie: $P(\tilde{C}_{k})=\frac{c_{2}}{n_{2}}$
Zachodza nastepujace wlasnosci:\\
$P(B_{k}\mid D_{k})=P(\tilde{B}_{k-1})$\\
$P(B_{k}\mid D_{k}^{-1})=P(B_{k-1})$, bo jesli nie wylosowalismy swiezo dodanej to niejako robimy to samo co w losowaniu k-1\\
$P(D_{k})=\frac{1}{n_{1}}$, bo z n-kul w urnie pierwszej tylko jedna jest ta swiezo dodana\\
$P(D_{k}^{-1})=1-P(D_{k})=\frac{n_{1}-1}{n_{1}}$\\
Ze wzoru na prawdopodobienstwo warunkowe:\\
$P(B_k)=P(B_{k}\mid D_{k})P(D_{k})+P(B_{k}\mid D_{k}^{-1})P(D_{k}^{-1})$\\
Po podstawieniu powyzszych:\\
$P(B_k)=P(\tilde{B}_{k-1})\frac{1}{n_{1}}+P(B_{k-1})\frac{n_{1}-1}{n_{1}}=\frac{b_{2}}{n_{2}n_{1}}+P(B_{k-1})\frac{n_{1}-1}{n_{1}}$\\
Dostalismy wzor rekurencyjny. Po rozwiazaniu go dla warunku poczatkowego: $P(B_{1})=\frac{b_{1}}{n_{1}}$ Otrzymujemy ostatecznie wzor:\\
$P(B_k)=\left ( \frac{n_{1}-1}{n_{1}} \right )^{k-1}\cdot \left ( \frac{b_{1}}{n_{1}}-\frac{b_{2}}{n_{2}} \right )+\frac{b_{2}}{n_{2}}$\\
Mozna sprawdzic poprawnosc indukcyjnie.\\
Odpowiedz do podpunktu b) to $P(B_{n})$
\\
a)
Podstawiamy do wzoru $n_{1}=n_{2}=b_{1}=n$, $b_{2}=0$. Odpowiedz to:
$\left ( \frac{n-1}{n} \right )^{n-1}$

\medskip
\noindent{\bf Zad. 6 (IS)}
\medskip

Klasyczny problem Monty Halla psychopatycznie sprowadzony do zabijania wiezniow...

Musimy udowodnić, że niezależnie od tego co mówi więzień, szansa na przezycie to zawsze $\frac{1}{3}$

Zalozmy ze A bedzie oznaczało sytuacje, kiedy wiezien 1 jest ulaskawiony, a B bedzie oznaczac sytuacje kiedy straznik powie, że inny wiezien bedzie skazany.


Z twierdzenia Bayesa:

$\P(A\mid B)=\frac {P(B\mid A)\P(A)}{P(B\mid A)P(A)+P(B\mid A^{'})P(A^{'})}.$



$P(A)$ to prawdopodobienstwo z jakim ulaskawiony jest pierwszy wiezen bez zadnych informacji - $\frac{1}{3}$

$P(A')$ to prawdopodobiensto z jakim pierwszy wiezien umiera (bez zadnych informacji) - $\frac{2}{3}$

$P(B \mid A)$ to prawdopodobienstwo ze straznik powie, ze inny wiezen umiera, jesli pierwszy przezywa. Straznik zawsze mowi ze inny umiera - prawdopodobienstwo wynosi 1

$P(B \mid A')$ to prawdopodobienstwo ze straznik powie, ze inny wiezen umiera, jesli pierwszy tez umrze. Straznik zawsze mowi ze inny umiera - prawdopodobienstwo wynosi 1

Podstawiajac wszystko do wzoru:

\[P(A | B) = \frac{1 \cdot \frac{1}{3}}{1 \cdot \frac{1}{3} + 1 \cdot \frac{2}{3}} = \frac{1}{3}  \]
 
\medskip
\noindent{\bf Zad. 7 (MM)}
\medskip
Prawdopodobienstwo ze informacja nie dojdze do celu:\\
$P(E \cap (A \cup B \cup (C \cap D)))=P(E)P(A \cup B \cup (C \cap D))=\\=P(E)(P(A)+P(B)+P(C \cap D)-P(A \cap B)-P(A\cap (C \cap D))-P(B\cap (C \cap D))+P(A \cap B \cap (C \cap D)))=\\=P(E)(P(A)+P(B)+P(C)P(D)-P(A)P(B)-P(A)P(C)P(D)-P(B)P(C)P(D)+P(A)P(B)P(C)P(D))=0.01\cdot (0.05+0.1+0.05\cdot 0.01-0.05 \cdot 0.1-0.05\cdot 0.05\cdot 0.01-0.1\cdot 0.05\cdot 0.01+0.05\cdot 0.1\cdot 0.05\cdot 0.01)=0.001454275$
Odpowiedz to: 1-0.001454275=0.998545725

\medskip
\noindent{\bf Zad. 8 (MK)}
\medskip

A - moduł A zawiera błędy

B - moduł B zawiera błędy

AB - oba moduły zawierają błędy

$P(A) = 0.2$
$P(B) = 0.4$

wydarzenia A i B są od siebie niezależne

$P(AB) = P(A) \cdot P(B) = 0.2 \cdot 0.4 = 0.08$

AwA - awaria wywołana błędem w samym A

AwB - awaria wywołana błędem w samym B

AwAB - awaria wywołana wadliwościa obu modułów

$P(AwA) = 0.5$

$P(AwB) = 0.8$

$P(AwAB) = 0.9$

moduł A był wadliwy i wywołał awarię:

$P(A) \cdot P(AwA) = 0.1$

moduł B był wadliwy i wywołał awarię:

$P(B) \cdot P(AwB) = 0.32$

oba moduły były wadliwe i nastąpiła awaria:

$ P(AB) \cdot P(AwAB) = 0.08 \cdot 0.9 = 0.072$

zatem:

$\frac{0.072}{0.1 + 0.32 + 0.072} = \frac{0.072}{0.492} = 0.146$

Odpowiedź: 0.146

\medskip


\noindent{\bf Zad. 9 (SS)} \\

1) Strategia: liczymy na to, że ostatnia karta to as pik. \\
Prawdopodobieństwo, że w ostatnia karta to as pik wynosi:$ \frac{1}{n}\quad$ \\
\\
2) Strategia: liczymy na to, że kolejna karta to as pik.
Więc:\\
$E_{1} =  \frac{1}{n + 1 - 1}\quad$ \\
$E_{2} = \frac{1}{n + 1 - 2}\quad$ \\
...\\
$E_{n} = \frac{1}{n + 1 - n}\quad$ \\ \\
Z reguły łańcuchowej: \\
$P(E_{1}E_{2}...E_{N}) = P(E_{1}) \cdot P(E_{2}| E_{1}) \cdot P(E_{3}|E_{1}E_{2}) \cdot ... \cdot P(E_{n}|E_{1}...E_{n-1})$\\
$P(E_{1}E_{2}...E_{N}) = \frac{1}{n} \cdot \frac{\frac{1}{n-1}}{\frac{1}{n}} \cdot ... \cdot \frac{\frac{n}{n}}{P(E_{1}...E_{n-1})} $

$ P(E_{1}E_{2}...E_{N}) < \frac{1}{n}$
Więc, lepiej przyjąć strategie 1).
\medskip

\medskip


\noindent{\bf Zad. 10 (SS)} \\
Wybieramy k mężczyzn, do których trafi ich kapulesz (k punktów stałych permutacji): ${n}\choose{k}$\\
Wybieramy n - k mężczyzn do których nie trafi ich kapelusz:\\ $\sum_{i=0}^{n-k} {{n-k}\choose{i}}  \cdot (n-k-i)! \cdot (-1)^{i-1} $

P =  $  \frac{{{n}\choose{k}} \cdot{\sum_{i=0}^{n-k} {{n-k}\choose{i}}  \cdot (n-k-i)! \cdot (-1)^{i-1}}  }{n!} $




\medskip
\end{document}