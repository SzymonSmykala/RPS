\documentclass[fleqn]{article}

\usepackage[T1]{fontenc}

\usepackage{amscd}
\usepackage{amsfonts}
\usepackage{latexsym}
\usepackage{graphicx}
\usepackage{amsthm}

\usepackage[utf8]{inputenc}
\usepackage[document]{ragged2e}
\usepackage{geometry}
\usepackage{amsmath}
\usepackage{amsfonts}
\usepackage{amssymb}
\usepackage{mathtools}
\usepackage{cancel}
\usepackage{polski}
\usepackage{calligra}
\usepackage{enumerate}

\def\B{\mathcal{B}}
\def\E{\mathcal{E}}
\def\F{\mathcal{F}}
\def\G{\mathcal{G}}
\def\O{\mathcal{O}}
\def\Q{\mathcal{Q}}
\def\P{\mathcal{P}}
\def\C{\mathcal{C}}
\def\U{\mathcal{U}}
\def\uni{\mathrm{unif}}
\def\Code{\mathrm{Code}}
\def\diag{\mathrm{diag}}
\def\Mah{\mathrm{Mah}}
\def\N{\mathbb{N}}
\def\Z{\mathbb{Z}}
\def\R{\mathbb{R}}
\def\Nor{\mathcal{N}}
\def\I{\mathrm{I}}
\def\f{\mathrm{f}}
\def\m{\mathrm{m}}

% definicje przydatnych potem komend

\def\C{{\mathbb C}}
\def\Z{{\mathbb Z}}
\def\N{{\mathbb N}}
\def\R{{\mathbb R}}
\def\Q{{\mathbb Q}}

\def\e{\varepsilon}
\def\w{\omega}
\def\dom{\mathrm{dom}}

\begin{document}

\title{Zadania z Rachunku Prawdopodobieństwa \\ Ćwiczenia .. z dnia ...}
\date{}
\medskip
\noindent{\bf Zad. 1} 
\medskip


\noindent{\bf Zad. 2} 
\medskip

\noindent{\bf Zad. 4 (SS)} 
\medskip

A - liczby podzielne przez 2 \\
B - liczby podzielne przez 3 \\
C - liczby podzielne przez 5 \\
\[|A| = 150\]
\[|B| = 100\]
\[|C| = 60 \]
\[|A \cap B| = 50\]
\[|A \cap C| = 30\]
\[|B \cap C| = 20\]
\[|A \cap B \cap C| = 10\]
\[|A \cup B \cup C| = |A| + |B| + |C| - (|A \cap B|  + |A \cap C| + |B \cap C|) + |A \cap B \cap C| = 220 \]

\noindent{\bf Zad. 5 (KA)}
\medskip

a) Ile jest możliwych różnych ścieżek tego punktu?

\[(PGPPG...) \qquad 
(\underbrace{PPP...P}_{\displaystyle\text{8 razy}}
\underbrace{GGG...G}_{\displaystyle\text{5 razy}})\]

W każdej trasie jest 8 kroków w prawo i 5 kroków do góry, łącznie 13 kroków.

\[{13 \choose 8} \cdot {5 \choose 5}
={13 \choose 8} \cdot 1={13 \choose 8}
= \frac{13!}{8!5!}
= \frac{9 \cdot \cancelto{2}{10} \cdot 11 \cdot \cancelto{1}{12} \cdot 13}
{1 \cdot 2 \cdot \cancel{3} \cdot \cancel{4} \cdot \cancel{5}}
= \frac{2574}{2}=1287\]

Odp. 1287 ścieżek.

\vspace{1em}

b) Ile jest ścieżek z a) przechodzących przez punkt $(5, 2)$?

\[(PGPPG...) \qquad 
(\underbrace{PPP...P}_{\displaystyle\text{5 razy}}
\underbrace{GGG...G}_{\displaystyle\text{2 razy}})\]
\[{7 \choose 5} \cdot {2 \choose 2}
= {7 \choose 5} \cdot 1
= {7 \choose 5}
= \frac{7!}{5!2!}
= \frac{\cancel{5!} \cdot 6 \cdot 7}{\cancel{5!} \cdot 1 \cdot 2}
= \frac{42}{2}=21\]

\[(PGPPG...) \qquad 
(\underbrace{PPP...P}_{\displaystyle\text{3 razy}}\underbrace{GGG...G}_{\displaystyle\text{3 razy}})\]
\[{6 \choose 3} \cdot {3 \choose 3}
={6 \choose 3} \cdot 1
={6 \choose 3}
=\frac{6!}{3!3!}
=\frac{\cancel{3!} \cdot 4 \cdot 5 \cdot \cancel{6}}
{\cancel{3!} \cdot 1 \cdot \cancel{2} \cdot \cancel{3}}
= 20\]

\[21  \cdot  20 = 420\]

Odp. 420 ścieżek.
\medskip

\noindent{\bf Zad. 6 (JK)} 
\medskip
\[{n\choose k}\]
\medskip

\noindent{\bf Zad. 8 (SS)} 
\medskip
\[\frac{6!}{4! \cdot 2!}\]
\medskip

\noindent{\bf Zad. 12 (SS)} 
\medskip
\[{5\choose 2} {6\choose 2} {4\choose 3}\]

\noindent{\bf Zad. 14 (JK)} 
\medskip
\[\frac{n!}{(n - r)!}\]
\medskip

\end{document}
