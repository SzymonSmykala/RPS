\documentclass[fleqn]{article}

\usepackage[T1]{fontenc}

\usepackage{amscd}
\usepackage{amsfonts}
\usepackage{latexsym}
\usepackage{graphicx}
\usepackage{amsthm}

\usepackage[utf8]{inputenc}
\usepackage[document]{ragged2e}
\usepackage{geometry}
\usepackage{amsmath}
\usepackage{amsfonts}
\usepackage{amssymb}
\usepackage{mathtools}
\usepackage{cancel}
\usepackage{polski}
\usepackage{calligra}
\usepackage{enumerate}

\def\B{\mathcal{B}}
\def\E{\mathcal{E}}
\def\F{\mathcal{F}}
\def\G{\mathcal{G}}
\def\O{\mathcal{O}}
\def\Q{\mathcal{Q}}
\def\P{\mathcal{P}}
\def\C{\mathcal{C}}
\def\U{\mathcal{U}}
\def\uni{\mathrm{unif}}
\def\Code{\mathrm{Code}}
\def\diag{\mathrm{diag}}
\def\Mah{\mathrm{Mah}}
\def\N{\mathbb{N}}
\def\Z{\mathbb{Z}}
\def\R{\mathbb{R}}
\def\Nor{\mathcal{N}}
\def\I{\mathrm{I}}
\def\f{\mathrm{f}}
\def\m{\mathrm{m}}

% definicje przydatnych potem komend

\def\C{{\mathbb C}}
\def\Z{{\mathbb Z}}
\def\N{{\mathbb N}}
\def\R{{\mathbb R}}
\def\Q{{\mathbb Q}}

\def\e{\varepsilon}
\def\w{\omega}
\def\dom{\mathrm{dom}}

\begin{document}

\title{Zadania z Rachunku Prawdopodobieństwa \\ Ćwiczenia .. z dnia ...}
\date{}
\medskip
\noindent{\bf Zad. 1 (KO)}
\medskip

Ilosc liter: 32. \\
Ilosc liter nadających sie dla pierwszej litery imienia/nazwiska: 25. \\
Zakładamy ze dla drugiej litery nadaje sie 32. \\
Ilosc roznych par imie-nazwisko, które maja dowolne pierwsze dwie litery: (25 \cdot 32) \cdot (25 \cdot 32) = 640,000. \\
Jesli w Krakowie mieszka przynajmniej 640,001 osób, to z podstawowej zadady zliczania wynika, ze przynajmniej 2 osoby maja 2 takie same pierwsze litery imienia i nazwiska. \\
Podobno w Krakowie jest 754,056 mieszkanców. \\
\medskip
\noindent{\bf Zad. 2 (VB)}
\medskip

Zakładamy że mamy grupę n ludzi.
Prawdopodobieństwo że dwie dowolne osoby mają wspolne urodziny wynosi $\frac{1}{365}$.\\
Prawdopodobieństwo że trzecia osoba ma wspolne urodziny z jedną z dwóch popzednich wynosi $\frac{2}{365}$.\\
Czyli dla n osob prawdopodobieństwo ze przynajmniej 2 osoby maja wspolne urodziny wynosi
\[P = \frac{1}{365} + \frac{2}{365} + \frac{3}{365} + ... + \frac{n - 1}{365} = \frac{n\cdot(n-1)}{2\cdot365}\]
Jesli P > 50\%, to n > 19.
\medskip

\noindent{\bf Zad. 3 (IS)}
\medskip

Mamy za zadanie wybrać k elementów spośród n elementowego zbioru. Pierwszy element możemy wybrać na n sposobów, drugi na (n - 1) sposobów, ... , i ostatni na (n - k - 1) sposobów. W taki sposób otrzymamy liczbę wszystkich permutacji zbioru k, z racji tego musimy wynik podzielić przez liczbę permutacji zbioru k - k!.\\
Otrzymujemy wynik:\\
\[\frac{n\cdot(n-1)\cdot...\cdot(n-k-1)}{k!} = \frac{n!}{k!\cdot(n-k)!}={n \choose k}\]

\medskip


\noindent{\bf Zad. 4 (SS)}
\medskip

A - liczby podzielne przez 2 ze zbioru \{1,2,3...300\}\\
B - liczby podzielne przez 3 ze zbioru \{1,2,3...300\}\\
C - liczby podzielne przez 5 ze zbioru \{1,2,3...300\}\\
\[|A| = 150\]
\[|B| = 100\]
\[|C| = 60 \]
\[|A \cap B| = 50\]
\[|A \cap C| = 30\]
\[|B \cap C| = 20\]
\[|A \cap B \cap C| = 10\]
\[|A \cup B \cup C| = |A| + |B| + |C| - (|A \cap B|  + |A \cap C| + |B \cap C|) + |A \cap B \cap C| = 220 \]

\noindent{\bf Zad. 5 (KA)}
\medskip

a) Ile jest możliwych różnych ścieżek tego punktu?

\[(PGPPG...) \qquad
(\underbrace{PPP...P}_{\displaystyle\text{8 razy}}
\underbrace{GGG...G}_{\displaystyle\text{5 razy}})\]

W każdej trasie jest 8 kroków w prawo i 5 kroków do góry, łącznie 13 kroków.

\[{13 \choose 8} \cdot {5 \choose 5}
={13 \choose 8} \cdot 1={13 \choose 8}
= \frac{13!}{8!5!}
= \frac{9 \cdot \cancelto{2}{10} \cdot 11 \cdot \cancelto{1}{12} \cdot 13}
{1 \cdot 2 \cdot \cancel{3} \cdot \cancel{4} \cdot \cancel{5}}
= \frac{2574}{2}=1287\]

Odp. 1287 ścieżek.

\vspace{1em}

b) Ile jest ścieżek z a) przechodzących przez punkt $(5, 2)$?

\[(PGPPG...) \qquad
(\underbrace{PPP...P}_{\displaystyle\text{5 razy}}
\underbrace{GGG...G}_{\displaystyle\text{2 razy}})\]
\[{7 \choose 5} \cdot {2 \choose 2}
= {7 \choose 5} \cdot 1
= {7 \choose 5}
= \frac{7!}{5!2!}
= \frac{\cancel{5!} \cdot 6 \cdot 7}{\cancel{5!} \cdot 1 \cdot 2}
= \frac{42}{2}=21\]

\[(PGPPG...) \qquad
(\underbrace{PPP...P}_{\displaystyle\text{3 razy}}\underbrace{GGG...G}_{\displaystyle\text{3 razy}})\]
\[{6 \choose 3} \cdot {3 \choose 3}
={6 \choose 3} \cdot 1
={6 \choose 3}
=\frac{6!}{3!3!}
=\frac{\cancel{3!} \cdot 4 \cdot 5 \cdot \cancel{6}}
{\cancel{3!} \cdot 1 \cdot \cancel{2} \cdot \cancel{3}}
= 20\]

\[21  \cdot  20 = 420\]

Odp. 420 ścieżek.
\medskip

\noindent{\bf Zad. 6 (JK)}
\medskip
\[{n\choose k}\]
\medskip

\noindent{\bf Zad. 7 (VB)}
\medskip

S = \big\{1, 2, .., 20 \big\}. \\
S' = S - \big\{1, 2, 3, 4, 5 \big\} = \big\{6, 7, .., 20 \big\}. \\
Ilość wszystkich 4-elementowych podzbiorów S wynosi ${20 \choose 4}$. \\
Ilość wszystkich 4-elementowych podzbiorów S' wynosi ${15 \choose 4}$. Są to podzbiory niezawierające zadnego elementu ze zbioru \big\{1, 2, 3, 4, 5 \big\}.\\
Stąd ilosc wszystkich podzbiorów S zawierających co najmniej jeden element z \big\{1, 2, 3, 4, 5 \big\} wynosi ${20 \choose 4}$ - ${15 \choose 4}$.\\

\medskip

\noindent{\bf Zad. 8 (SS)}
\medskip \\
Razem 6 liter. P powtarza się 3 razy, E powtarza się 2 razy.\\
\[\frac{6!}{3! \cdot 2!}\]
\medskip

\noindent{\bf Zad. 9 (VB)}
\medskip

m wadliwych anten ustawiamy w taki sposob, żeby między każde dwie ustawić przynajmniej jedną działającą. Czyli $(n - m)$ działających anten ukladamy na $(m - 1)$ miejscach. \\
Jest to równe liczbie surjekcji z $\big\{1, 2, .., (n-m) \big\}$ do $\big\{1, 2, .., (m-1) \big\}$. \\
\[(m-1)! \cdot S((n-m), (m-1)\]

\medskip

\noindent{\bf Zad. 10 (KO)}
\medskip

a) Są 4 mecze, kazdy ma 2 mozliwe wyniki. Czyli $2^4$.\\
b) W trzech turach łącznie jest 7 meczy, kazdy po 2 możliwe wyniki, czyli $2^7$

\medskip

\noindent{\bf Zad. 11 (VB)}
\medskip
\[x_1 + x_2 + ... + x_r = n \]
Jest równe liczbie surjekcji z $\big\{1, 2, .., n \big\}$ do $\big\{1, 2, .., r \big\}$. \\
\[r! \cdot S(n, r)\]
\medskip

\noindent{\bf Zad. 12 (SS)}
\medskip \\
Wybieramy 2 z 5 (PiS), 2 z 6 (PO) i 3 z 4 (.N) \\
\[{5\choose 2} {6\choose 2} {4\choose 3}\]

\noindent{\bf Zad. 13 (IS)}
\medskip
\[{n\choose k} & & \text{ to jest ilosc sposobow wyboru k elementowego podzbioru ze zbioru n elementowego}\]
Zamiast wybierać elementy, które należą do zbioru mozemy rowniez wybrac elementy, ktore nie beda nalezec do zbioru k, ilosc takich sposobow jest rowna. Zatem musimy wybrac (n - k) elementów, które nie będa należec do zbioru k. Ilosc tych sposobow wynosi
\[{n \choose n-k}\]
Zatem zachodzi rownosc:
\[{n\choose k}={n \choose n-k}\]

\medskip

\noindent{\bf Zad. 14 (JK)}
\medskip
\[\frac{n!}{(n - r)!}\]
\medskip

\end{document}
