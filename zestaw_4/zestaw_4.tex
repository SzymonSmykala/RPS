\documentclass[fleqn]{article}

\usepackage[T1]{fontenc}

\usepackage{amscd}
\usepackage{amsfonts}
\usepackage{latexsym}
\usepackage{graphicx}
\usepackage{amsthm}

\usepackage[utf8]{inputenc}
\usepackage[document]{ragged2e}
\usepackage{geometry}
\usepackage{amsmath}
\usepackage{amsfonts}
\usepackage{amssymb}
\usepackage{mathtools}
\usepackage{cancel}
\usepackage{polski}
\usepackage{calligra}
\usepackage{enumerate}

\def\B{\mathcal{B}}
\def\E{\mathcal{E}}
\def\F{\mathcal{F}}
\def\G{\mathcal{G}}
\def\O{\mathcal{O}}
\def\Q{\mathcal{Q}}
\def\P{\mathcal{P}}
\def\C{\mathcal{C}}
\def\U{\mathcal{U}}
\def\uni{\mathrm{unif}}
\def\Code{\mathrm{Code}}
\def\diag{\mathrm{diag}}
\def\Mah{\mathrm{Mah}}
\def\N{\mathbb{N}}
\def\Z{\mathbb{Z}}
\def\R{\mathbb{R}}
\def\Nor{\mathcal{N}}
\def\I{\mathrm{I}}
\def\f{\mathrm{f}}
\def\m{\mathrm{m}}

% definicje przydatnych potem komend

\def\C{{\mathbb C}}
\def\Z{{\mathbb Z}}
\def\N{{\mathbb N}}
\def\R{{\mathbb R}}
\def\Q{{\mathbb Q}}

\def\e{\varepsilon}
\def\w{\omega}
\def\dom{\mathrm{dom}}

\begin{document}

\title{Zadania z Rachunku Prawdopodobieństwa \\ Ćwiczenia .. z dnia ...}
\date{}
\medskip
\noindent{\bf Zad. 1 (PGo)}
\medskip

a) $X\in\mathbb{N}_n$

b) Prawdopodobieństwa wynoszą:
\begin{itemize}
\item $P\left\lbrace X=1 \right\rbrace=p$
\item $P\left\lbrace X=2 \right\rbrace=\left(1-p\right)\cdot p$

...

\item $P\left\lbrace X=i \right\rbrace=\left(1-p\right)^{i-1}\cdot p$

...

\item $P\left\lbrace X=n-1 \right\rbrace=\left(1-p\right)^{n-2}\cdot p$
\item $P\left\lbrace X=n \right\rbrace=\left(1-p\right)^{n-1}$
\end{itemize}
c) Można to udowodnić indukcyjnie.

Dla $n=1$ - $P\left\lbrace X=1 \right\rbrace=1$

Dla $n=2$ - $P\left\lbrace X=1 \right\rbrace=p$, $P\left\lbrace X=2 \right\rbrace=1-p$. Ich suma jest równa 1.

Zakładamy, że dla $n=k$ suma prawdopodobieństw wynosi $\sum\limits_{j=1}^{k-1}\left(1-p\right)^{j-1}\cdot p+\left(1-p\right)^{k-1}=1$.
Zatem dla $n=k+1$ wynosi $\sum\limits_{j=1}^{k}\left(1-p\right)^{j-1}\cdot p+\left(1-p\right)^{k}=1+\left(1-p\right)^{k-1}\cdot \left( p-1\right)+\left(1-p\right)^{k}=1$.

\medskip
\noindent{\bf Zad. 2 (VB)}
\medskip

3 białe (+1\$), 3 czerwone(-1\$), 5 czarnych(+0\$). \\
Losujemy 3 kule, to żeby całkowita suma byla dodatnia, jest 3 opcji: \\
1) 1\$: 1b+2czar = ${3 \choose 1} \cdot {5 \choose 2}$;  2b+1czer = ${3 \choose 2} \cdot {3 \choose 1}$; \\

2) 2\$: 2b+1czar = ${3 \choose 2} \cdot {5 \choose 1}$; \\

3) 3\$: 3b = ${3 \choose 3}$; \\

Ilość wszystkich mozliwych kombinacji wynosi ${11 \choose 3}$. \\
Prawdopodobieństwo wygranej wynosi:
\[\frac{{3 \choose 1} \cdot {5 \choose 2} + {3 \choose 2} \cdot {3 \choose 1} + {3 \choose 2} \cdot {5 \choose 1} + 1}{{11 \choose 3}}\]

\medskip
\noindent{\bf Zad. 6 (Spisał: PGo)}
\medskip

$Var(X)=E(X^2)-E^2(X)$

$Var\left( aX+b\right) =E\left(\left( aX+b\right)^2\right) - E^2\left( aX+b \right)=E\left( a^2X^2+2abx+b^2\right) -\left(aE\left( X\right)+b\right)^2=a^2E\left(X^2\right)+2abE\left(X\right)-a^2E^2\left(X\right)-2abE\left(X\right)-b^2=a^2\cdot\left(E\left(X^2\right)-E^2\left(X\right)\right)=a^2Var\left(X\right)$

\medskip
\noindent{\bf Zad. 7 (Spisał: PGo)}
\medskip

$p$ - prawdopodobieństwo, że śrubka jest wadliwa

$n$ - ilość śrubek w pakiecie

$X$ - zmienna losowa oznaczająca ilość wadliwych śrubek

$P\left\lbrace X=0\right\rbrace={n\choose 0}\cdot p^0\cdot\left(1-p\right)^n=0,9044$

$P\left\lbrace X=1\right\rbrace={n\choose 1}\cdot p^1\cdot\left(1-p\right)^{n-1}=0,0914$

Prawdopodobieństwo tego, że nie będzie zwrotu pieniędzy wynosi: $1-\left(P\left\lbrace X=0\right\rbrace + P\left\lbrace X=1\right\rbrace \right)=0,0042$

\medskip
\noindent{\bf Zad. 8 (Spisał: PGo)}
\medskip

$\sum_{i=0}^{\infty}P\left\lbrace X=i\right\rbrace=e^{-\lambda}\cdot\sum_{i=0}^{\infty}\frac{\lambda^i}{i!}=e^{-\lambda}\cdot e^\lambda=1$

\end{document}
