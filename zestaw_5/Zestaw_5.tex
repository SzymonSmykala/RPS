\documentclass[fleqn]{article}

\usepackage[T1]{fontenc}

\usepackage{amscd}
\usepackage{amsfonts}
\usepackage{latexsym}
\usepackage{graphicx}
\usepackage{amsthm}

\usepackage[utf8]{inputenc}
\usepackage{geometry}
\usepackage{amsmath}
\usepackage{amsfonts}
\usepackage{amssymb}
\usepackage{mathtools}
\usepackage{cancel}
\usepackage{calligra}
\usepackage{enumerate}

\def\B{\mathcal{B}}
\def\E{\mathcal{E}}
\def\F{\mathcal{F}}
\def\G{\mathcal{G}}
\def\O{\mathcal{O}}
\def\Q{\mathcal{Q}}
\def\P{\mathcal{P}}
\def\C{\mathcal{C}}
\def\U{\mathcal{U}}
\def\uni{\mathrm{unif}}
\def\Code{\mathrm{Code}}
\def\diag{\mathrm{diag}}
\def\Mah{\mathrm{Mah}}
\def\N{\mathbb{N}}
\def\Z{\mathbb{Z}}
\def\R{\mathbb{R}}
\def\Nor{\mathcal{N}}
\def\I{\mathrm{I}}
\def\f{\mathrm{f}}
\def\m{\mathrm{m}}

% definicje przydatnych potem komend

\def\C{{\mathbb C}}
\def\Z{{\mathbb Z}}
\def\N{{\mathbb N}}
\def\R{{\mathbb R}}
\def\Q{{\mathbb Q}}

\def\e{\varepsilon}
\def\w{\omega}
\def\dom{\mathrm{dom}}

\begin{document}

\title{Zestaw 5}
\date{}
\medskip
\noindent{\bf Zad. 2 (MM)}
\medskip
\\
Ogolniej zamiast liczyc funkcje gestosci dla $2X$ policzmy dla $aX$ gdzie $a\neq 0$.\\
zachodzi: $\frac{\partial F_{X} }{\partial c}(c_{0})=f_{X}(c_{0})$ dla dowolnej zmiennej losowej ciąglej X. Niech $ a>0$.\\
$F_{aX}(c)=P(aX< c)=P(X<\frac{c}{a})=F_{X}(\frac{c}{a})$ Zgodnie z wczesniejszym wzorem:\\
$f_{aX}(c_0)=\frac{\partial F_{aX} }{\partial c}(c_{0})=\frac{\partial F_{X} }{\partial c}(\frac{c_{0}}{a})=f_{X}(\frac{c_{0}}{a})\frac{\partial (\frac{c_{0}}{a}) }{\partial c}=\frac{1}{a}f_{X}(\frac{c_{0}}{a})$\\
Anologicznie liczymy dla $a<0$ (Trzeba odwrocic nierownosc w prawdopodobienstwie). Wychodzi: $f_{aX}(c_0)=-\frac{1}{a}f_{X}(\frac{c_{0}}{a})$\\ Zatem ostateczny wzór to:\\
 $f_{aX}(c_0)=\frac{1}{|a|}f_{X}(\frac{c_{0}}{a})$


\medskip

\noindent{\bf Zad. 5 (SS)}
\medskip

$$
f(x)=
\begin{cases}
0 \quad x \leq \alpha,\\
\dfrac{1}{b-a} \quad \alpha < x < \beta\\
0 \quad beta \leq x
\end{cases}
$$

$$ EX = \int\limits_{-\infty}^{\infty} xf(x)dx = 
 \int\limits_{\alpha}^{\beta} xf(x)dx = \int\limits_{\alpha}^{\beta} \frac{x}{b-a} = \frac{a + b}{2} \\ $$
VarX = EX^2 - EX \\ $$  $$
$$
VarX = \int\limits_{\alpha}^{\beta} x^2 f(x)dx - \int\limits_{\alpha}^{\beta} xf(x) dx= \frac{(b-a)^2}{12}
$$

\noindent{\bf Zad. 6 (MM)}
\medskip
\\
Zmienna losowa Y o rozkladzie jednostajnym bedzie przyjmowac wartosci z przedziału $(0,30)$ i oznaczac to bedzie moment w ktorym pasazer przyszedl na przystanek (0 to 7:00, 15 to 7:15 itd)\\
a)\\
Szukana wartosc to: $P(((Y+5>15)\cap(Y<15))\cup((Y>15)\cap(Y+5>30)) )$ \\
$ P(((Y+5>15)\cap(Y<15))\cup((Y>15)\cap(Y+5>30)) )=P(Y\in(10,15))+P(Y>25)=\\
=F_{Y}(15)-F_{Y}(10)+1-F_{Y}(25)=\frac{15}{30}-\frac{10}{30}+1-\frac{25}{30}=\frac{1}{3}$\\
b)\\
Szukana wartosc to: $1-P(((Y+10>15)\cap(Y<15))\cup((Y>15)\cap(Y+10>30)) )$ \\
$ 1-P(((Y+10>15)\cap(Y<15))\cup((Y>15)\cap(Y+10>30)) )=1-(P(Y\in(5,15))+P(Y>20))=\\
=1-(F_{Y}(15)-F_{Y}(5)+1-F_{Y}(20))=1-(\frac{15}{30}-\frac{5}{30}+1-\frac{20}{30})=\frac{1}{3}$\\


\medskip

\end{document}