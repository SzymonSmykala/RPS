\documentclass[fleqn]{article}

\usepackage[T1]{fontenc}

\usepackage{amscd}
\usepackage{amsfonts}
\usepackage{latexsym}
\usepackage{graphicx}
\usepackage{amsthm}

\usepackage[utf8]{inputenc}
\usepackage[document]{ragged2e}
\usepackage{geometry}
\usepackage{amsmath}
\usepackage{amsfonts}
\usepackage{amssymb}
\usepackage{mathtools}
\usepackage{cancel}
\usepackage{polski}
\usepackage{calligra}
\usepackage{enumerate}

\def\B{\mathcal{B}}
\def\E{\mathcal{E}}
\def\F{\mathcal{F}}
\def\G{\mathcal{G}}
\def\O{\mathcal{O}}
\def\Q{\mathcal{Q}}
\def\P{\mathcal{P}}
\def\C{\mathcal{C}}
\def\U{\mathcal{U}}
\def\uni{\mathrm{unif}}
\def\Code{\mathrm{Code}}
\def\diag{\mathrm{diag}}
\def\Mah{\mathrm{Mah}}
\def\N{\mathbb{N}}
\def\Z{\mathbb{Z}}
\def\R{\mathbb{R}}
\def\Nor{\mathcal{N}}
\def\I{\mathrm{I}}
\def\f{\mathrm{f}}
\def\m{\mathrm{m}}

% definicje przydatnych potem komend

\def\C{{\mathbb C}}
\def\Z{{\mathbb Z}}
\def\N{{\mathbb N}}
\def\R{{\mathbb R}}
\def\Q{{\mathbb Q}}

\def\e{\varepsilon}
\def\w{\omega}
\def\dom{\mathrm{dom}}

\begin{document}

\title{Zadania z Rachunku Prawdopodobieństwa \\ Ćwiczenia .. z dnia ...}
\date{}
\medskip
\noindent{\bf Zad. 1 (PGo)}
\medskip

Możemy wylosować k kul spośród n kul na tyle sposobów:

$|\Omega|={n\choose k}$

Wyróżnioną kulę możemy wybrać na 1 sposób, pozostałe na ${n-1\choose k-1}$ sposobów.

$P=\frac{{n-1\choose k-1}}{{n\choose k}}=\frac{\frac{(n-1)!}{(k-1)!\cdot(n-k)!}}{\frac{n!}{k!\cdot(n-k)!}}=\frac{k}{n}$

\medskip
\noindent{\bf Zad. 2 (PGo)}
\medskip

Losowy dobór oznacza tyle możliwości umieszczenia w pokojach:

$|\Omega|={40\choose 2}\cdot{38\choose 2}\cdot...\cdot{2\choose 2}=\frac{40!}{2^{20}}$

Ilość umieszczeń w pokojach osobno księży, osobno zakonnic:

$|A|=\left({20\choose 2}\cdot{18\choose 2}\cdot...\cdot{2\choose 2}\right)^{2}\cdot\frac{20!}{10!^2}=\frac{20!^3}{2^{20}\cdot10!^2}$

Czynnik $\frac{20!}{10!^2}$ odpowiada za "wymieszanie pokoi".

$P(A)=\frac{20!^3}{40!\cdot10!^2}\approx0,00013\%$

Ilość takich umieszczeń w pokojach, że występuje dokładnie 2i mieszanych par:

$|B|=\left({20-2i\choose 2}\cdot{18-2i\choose 2}\cdot...\cdot{2\choose 2}\right)^{2}\cdot\left({4i\choose 2}\cdot{4i-2\choose 2}\cdot...\cdot{2\choose 2}\right)\cdot\frac{10!}{2i!\cdot(10-i)!^2}=\frac{(20-2i)!^2\cdot4i!\cdot10!}{2^{20}\cdot2i!\cdot(10-i)!^2}$

$P(B)=\frac{\frac{(20-2i)!^2\cdot4i!\cdot10!}{2^{20}\cdot2i!\cdot(10-i)!^2}}{\frac{40!}{2^{20}}}=\frac{(20-2i)!^2\cdot4i!\cdot10!}{40!\cdot2i!\cdot(10-i)!^2}$

\medskip
\noindent{\bf Zad. 5 (PGa)}
\medskip

x - godzina, w której zgłasza się pierwszy proces

y - godzina, w której zgłasza się drugi proces

$x,y \in [10,12]$

$|x-y|\le\frac{10}{60}$ - warunek zajścia awarii (zdarzenia A)

Skalujemy przedział do [0,1] i rozwiązujemy powyższą nierówność:

$x-y\le\frac{10}{120} \ \wedge \ x-y\ge-\frac{10}{120}$

$x\le y+\frac{1}{12} \ \wedge \ x\ge y-\frac{1}{12}$

W tym wypadku, do uzyskania prawdopodobieństwa zajścia awarii wystarczy policzyć pole figury znajdującej się między powyższymi prostymi ograniczonymi przez kwadrat zbudowany na podstawie przeskalowanego przedziału czasowego.

Najłatwiej zrobić to odejmując od pola kwadratu (1x1) dwa trójkąty stworzone przez te proste

$P(A)=1\cdot1-2\cdot(\frac{1}{2}\cdot(1-\frac{1}{12})^{2})=1-2\cdot\frac{1}{2}\cdot(\frac{11}{12})^{2}=1-\frac{121}{144}=\frac{23}{144}$

\end{document}
